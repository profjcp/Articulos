% Plantilla latex para publicación de artículo científico en la revista "SOE"
%
% @autor {Juan Carlos Peinado Pereira}
% @correo {juanpeinado@uagrm.edu.bo}
% @año 2025
% @versión 1.0
% @licencia CC BY 4.0
%
% Se agradecen sus comentarios, contribuciones, reporte de bugs y la difusión de esta plantilla.

% Tipo de documento ytamaño de letra
\documentclass[10pt]{article}

% Ajuste de lenguaje
\usepackage[spanish, english]{babel}

% Tamaño de página y margenes
\usepackage[letterpaper, top=2.5cm,bottom=3.5cm, left=3cm, right=2.5cm, marginparwidth=1.75cm]{geometry}

% Permite manejo de ecuaciones y símbolos matemáticos
\usepackage{amsmath, amssymb, amsthm, amsfonts}

% Permite el manejo de caracteres especiales en el español
\usepackage[utf8]{inputenc} 
\usepackage[T1]{fontenc}

% Permite el manejo de citas bibliográficas
\usepackage[backend=biber, style=apa, natbib=true]{biblatex}
\usepackage{csquotes}
% Permite el manejo de citas bibliográficas
\addbibresource{referencias.bib}
% Permite el manejo de citas bibliográficas
%\usepackage{cite}

% Permite manejo de la norma APA Apacite referencias bibliográficas
%\usepackage{apacite}

% Paquete para agregar imágenes
\usepackage{graphicx}

% Fuente principal del texto: Times
\usepackage{mathptmx}

% Permite el manejo de caracteres especiales en el español
\usepackage[utf8]{inputenc} 
\usepackage[T1]{fontenc}

% Subrayado de textos
\usepackage[normalem]{ulem}

% Permite justificar el texto
\usepackage{ragged2e}

% Personalizar los encabezados y pies de página
\usepackage{fancyhdr}
\pagestyle{fancy}
\setlength{\headheight}{36.2396pt}
\addtolength{\topmargin}{-24.2396pt}

% Ruta para imagenes
\graphicspath{{image/}}

% Ajusta la sangría
\setlength{\parindent}{1cm}

% Ajusta el espacio entre renglones
\setlength{\parskip}{0.5cm}

% Paquete para hipervinculos (debe ir después de cite y apacite)
\usepackage{hyperref}
\hypersetup{
    colorlinks=true,
    linkcolor=black,
    urlcolor=black,
    citecolor=black
}
\usepackage{listings} 
\usepackage{xcolor}
\definecolor{codegreen}{rgb}{0,0.6,0}
\definecolor{codegray}{rgb}{0.5,0.5,0.5}
\definecolor{codepurple}{rgb}{0.58,0,0.82}
\definecolor{backcolor}{rgb}{0.95,0.95,0.92}

\lstdefinestyle{mystyle}{
	backgroundcolor=\color{backcolor},
	commentstyle=\color{codegreen},
	keywordstyle=\color{magenta},
	numberstyle=\color{codegray},
	stringstyle=\color{codepurple},
	basicstyle=\ttfamily\footnotesize
}

% Headers config
% Cambiar Volumen, Numero y páginas
\lhead{
    Fronteras Tecnologicas, Febrero 20xx, Vol. x, No. x, páginas xx – xx\\Disponible en línea en \href{https://www.soe.uagrm.edu.bo/revistas/FTEC}{soe.uagrm.edu.bo/revistas/FTEC}\\ISSN: xx-xx, Postgrado de Ingenieria en Ciencias de la Computación y Telecomunicaciones, UAGRM
}
\rhead{
    \includegraphics[width=1.0cm]{logo}}
\headsep=60pt

% Información del artículo
\title{\Huge Latex una herramienta a disposición del investigador\\}
\author{\Large {1\textsuperscript{st}Juan Carlos Peinado Pereira}}
\date{\normalsize Posgrado SOE - UAGRM\\
\textit{jcpeinado@soe.uagrm.edu.bo}\\
Santa Cruz de la Sierra, Bolivia \\
https://orcid.org/0009-0009-9117-2441}


\begin{document}
\maketitle

\selectlanguage{spanish}
\begin{abstract}
    \textit{\normalsize Este artículo examina el uso de LaTeX como herramienta para la investigación, destacando sus ventajas sobre los procesadores de texto tradicionales. 
    Se enfoca en la precisión tipográfica, el manejo de matemáticas, la gestión de referencias y la estructura de documentos. 
    El autor argumenta que LaTeX mejora la calidad y eficiencia en la producción de documentos científicos, especialmente en ciencias de la computación. El texto presenta una plantilla base en LaTeX para investigadores, junto con una comparativa entre Word y LaTeX.
    Se concluye que LaTeX eleva el prestigio académico y facilita la difusión de la investigación, aunque requiere capacitación inicial.} 
    \vspace{0.5cm}

    \textbf{Palabras clave:} Procesador de texto, Herramientas de investigacion, Tipografía, Gestion referencial, LaTeX.
\end{abstract}


\selectlanguage{english}
\begin{abstract}
    \textit{\normalsize This article examines the use of LaTeX as a research tool, highlighting its advantages over traditional word processors. 
    It focuses on typographical accuracy, mathematics, reference management, and document structure. 
    The author argues that LaTeX improves the quality and efficiency in the production of scientific documents, especially in computer science. 
    The text presents a base LaTeX template for researchers, along with a comparison between Word and LaTeX. 
    It is concluded that LaTeX increases academic prestige and facilitates the dissemination of research, although it requires initial training.}    
    \vspace{0.5cm}

    \textbf{Keywords:} Word processor, Research tools, Typography, Reference management, LaTeX.
\end{abstract}

\selectlanguage{spanish}

    \section{Introducción}

LaTeX, creado por Donald Knuth, es un programa de composición tipográfica que se ha convertido en el estándar de facto para la publicación de artículos científicos y libros académicos en campos como matemáticas y física \cite{knuth1997art}. 
A diferencia de los procesadores de texto tradicionales, LaTeX ofrece precisión y control tipográfico, manejo excepcional de matemáticas y símbolos, gestión eficiente de referencias y citas, estructura y organización de documentos, y portabilidad y compatibilidad. 
Universidades de renombre como el MIT y Stanford, así como organizaciones científicas como la Sociedad Americana de Física (APS), utilizan LaTeX para sus publicaciones. Este estudio tiene como finalidad dar a conocer las ventajas de LaTeX a los investigadores, proporcionando una plantilla base para la unidad de posgrado SOE.\\
\textbf{¿Qué es Latex?}

TeX es un programa de composición tipográfica creado por Donald Knuth, originalmente para su obra maestra, The Art of Computer Programming. Toma un archivo de texto "plano" y lo convierte en un documento de alta calidad para imprimir o visualizar en pantalla. LaTeX es un sistema de macros construido sobre TeX que tiene como objetivo simplificar su uso y automatizar muchas tareas comunes de formato. Es el estándar de facto para revistas y libros académicos en varios campos, como matemáticas y física, y ofrece algunas de las mejores tipografías que el software libre tiene para ofrecer.\@cite{knuth1997art}

LaTeX se ha convertido en una herramienta esencial para investigadores en ciencias de la computación debido a una combinación de propiedades que lo distinguen de los procesadores de texto tradicionales. Estas son algunas de las principales razones por las que LaTeX es tan valioso en este campo:

\begin{enumerate}  
    \item Precisión y control tipográfico:
Calidad de la presentación: LaTeX ofrece un control preciso sobre la tipografía, el espaciado, la alineación y otros detalles de diseño, lo que resulta en documentos con una apariencia profesional y pulcra.
Énfasis en el contenido: Al separar el contenido del formato, LaTeX permite a los investigadores concentrarse en la sustancia de su trabajo, dejando que el sistema se encargue de la presentación. 
    \item Manejo excepcional de matemáticas y símbolos:
Notación matemática avanzada: LaTeX facilita la escritura de ecuaciones complejas, fórmulas y símbolos matemáticos con una claridad y precisión inigualables.
Estándar en la comunidad científica: LaTeX se ha convertido en el estándar para la publicación de artículos científicos y técnicos, especialmente en campos como las matemáticas, la física y la informática.
    \item Gestión eficiente de referencias y citas:
Automatización de referencias: LaTeX simplifica la gestión de referencias bibliográficas, citas y notas al pie de página, lo que ahorra tiempo y reduce errores.
Consistencia y precisión: LaTeX garantiza la consistencia y precisión en el formato de las referencias, lo cual es crucial para la credibilidad de la investigación.
	\item Estructura y organización de documentos:
Organización jerárquica: LaTeX facilita la organización de documentos extensos en secciones, subsecciones, capítulos y otras estructuras jerárquicas.
Generación automática de índices y tablas de contenido: LaTeX puede generar automáticamente índices, tablas de contenido y listas de figuras y tablas, lo que facilita la navegación y la consulta de documentos.
	\item Portabilidad y compatibilidad:Independencia de plataforma: Los documentos LaTeX son independientes de la plataforma, lo que significa que se pueden crear y visualizar en diferentes sistemas operativos sin perder calidad ni formato.
Formatos de salida versátiles: LaTeX permite generar documentos en diversos formatos, como PDF, PostScript y HTML, lo que facilita su distribución y publicación en diferentes medios.
	\item Comunidad y recursos:
Amplia comunidad de usuarios: LaTeX cuenta con una gran comunidad de usuarios y desarrolladores que ofrecen apoyo, recursos y paquetes adicionales para extender sus funcionalidades.
Documentación exhaustiva: LaTeX dispone de una documentación completa y detallada, así como de numerosos tutoriales y cursos en línea para aprender a utilizarlo.
\end{enumerate}  


%\begin{figure}[htbp]
%\centerline{\includegraphics{Figures/GenAiGra.png}}%
%\caption{Diagram of a typical GenAI application architecture.}
%\label{fig}
%\end{figure}

%\begin{figure}[ht]
% \centering
% \makebox[\linewidth][c]{\includegraphics[width=1\linewidth]{Figures/GenAIGra.png}}%
% \caption{Diagram of a typical GenAI application architecture}
% \label{fig-1}
%\end{figure}


\textbf{¿Quienes emplean Latex?}

Instituciones que utilizan LaTeX:

\begin{itemize}
	\item Universidades de renombre: Universidades líderes como el MIT, Stanford, Harvard, Oxford y Cambridge, entre muchas otras, utilizan LaTeX para la preparación de tesis, artículos y otros documentos académicos.
	\item Institutos de investigación: Institutos como el CERN, el Instituto Max Planck, el CNRS y el Instituto Nacional de Estándares y Tecnología (NIST) también emplean LaTeX para sus publicaciones.
	\item Organizaciones científicas: Organizaciones como la Sociedad Americana de Física (APS), la Asociación para la Maquinaria de la Computación (ACM) y la Sociedad de Matemáticas Aplicadas e Industriales (SIAM) utilizan LaTeX como estándar para sus publicaciones.
\end{itemize}

Publicaciones que utilizan LaTeX como estándar:

\begin{itemize}
	\item Revistas científicas: La mayoría de las revistas científicas de alto impacto en áreas como física, matemáticas, informática, ingeniería y ciencias de la computación requieren que los manuscritos se presenten en formato LaTeX.
	\item Actas de conferencias: Las actas de conferencias de prestigio en ciencias de la computación y otras áreas técnicas a menudo utilizan LaTeX para garantizar la uniformidad y calidad de los documentos.
	\item Libros académicos: Muchos libros de texto y obras de referencia en ciencias y matemáticas se publican utilizando LaTeX debido a su capacidad para manejar notaciones matemáticas complejas y producir documentos de alta calidad.
\end{itemize}

Fuentes de datos que emplean LaTeX:
\begin{itemize}
	\item Google Scholar: Google Scholar indexa artículos científicos y otros documentos académicos en una amplia variedad de disciplinas. Muchos de estos documentos están escritos en LaTeX, especialmente en áreas como matemáticas, física e informática.
	\item Scopus: Scopus es una base de datos bibliográfica de Elsevier que indexa una gran cantidad de literatura científica, incluyendo artículos de revistas, actas de conferencias y libros. Muchos de los documentos indexados en Scopus están escritos en LaTeX.
	\item Web of Science: Web of Science es otra base de datos bibliográfica ampliamente utilizada que indexa publicaciones científicas de diversas disciplinas. Al igual que Scopus, muchos de los documentos indexados en Web of Science están escritos en LaTeX.
\end{itemize}

\textbf{¿Esquema de compilacion de Latex?}\\
El siguiente esquema muestra el proceso de compilación de un documento LaTeX:
\begin{figure}[ht]
    \centering
    \makebox[\linewidth][c]{\includegraphics[width=0.5\linewidth]{LatexD.png}}%
    \caption{Esquema de compilacion de Latex}
    \label{fig:LatexD}
\textit{nota: En la figura se muestra el proceso de compilación de un documento LaTeX. El archivo fuente (.tex) se compila utilizando un motor LaTeX (pdfLaTeX, XeLaTeX, LuaLaTeX) para generar un archivo PDF. El proceso puede requerir varias pasadas para resolver referencias cruzadas, citas bibliográficas y otros elementos.}
\end{figure}

\newpage % Agrega o elimina estos saltos de página de acuerdo a tus necesidades.
\vspace{0.5cm}

    \section{Material y Metodo}


    El nombre de éste segundo título dependerá de cómo se haya estructurado el artículo.
    
    Se debe considerar una sección en la cual se presente la metodología a seguir, utilizando de preferencia un mapa conceptual (esquemas, diagrama a bloques, de flujo, etc.) para mostrar gráficamente la metodología utilizada. Algunas de las secciones que pudieran llevar la estructura del artículo son: Descripción del problema, Metodología, Modelación y Simulación, Diseño, Manufactura, Integración, Metodología, etc.

    \subsection{Diseño del estudio}
    Este estudio tiene como finalidad dar a conocer las numerosas ventajas que LaTeX ofrece a los investigadores, incluyendo su capacidad para generar documentos con una calidad tipográfica superior, su eficiente manejo de fórmulas matemáticas y símbolos complejos, y su automatización en la gestión de referencias y citas bibliográficas. Estas características son especialmente valiosas en el campo de la investigación en ciencias de la computación, donde la precisión y la presentación profesional son fundamentales, también se dotara de una plantilla base como apoyo a los investigadores de la unidad de posgrado SOE.

    Fases a tomar en cuenta para la elaboración de una plantilla base en Latex :
    
    Definir los requisitos
    \begin{enumerate}
     
        \item Tipo de documentos: Determina qué tipo de documentos se crearán con la plantilla (tesis, artículos, presentaciones, informes, etc.). Cada uno tendrá requisitos específicos de formato.
        \item Normativa de la SOE: Investiga si la unidad de posgrado tiene lineamientos de formato específicos (márgenes, tamaño de letra, espaciado, etc.).
    Estilo de citación: Define el estilo de citación requerido (APA, MLA, Chicago, etc.) y los paquetes de LaTeX necesarios para implementarlo.
    Elementos obligatorios: Identifica los elementos que deben estar presentes en todos los documentos (portada, índice, resumen, bibliografía, etc.).
    \end{enumerate}
    
    Estructura básica de la plantilla
    \begin{enumerate}
    
        \item Clase de documento: Elige la clase de documento adecuada (article, book, report, beamer, etc.).
        \item Paquetes esenciales: Incluye los paquetes básicos para el manejo de idiomas, codificación de caracteres, matemáticas, gráficos y otros elementos comunes.
        \item Definición de comandos: Crea comandos personalizados para abreviar tareas repetitivas (encabezados, pies de página, formatos de texto, etc.).
        \item Diseño de la portada: Diseña la estructura de la portada, incluyendo el logo de la SOE, el título del documento, el autor, la fecha y otros datos relevantes.
    Configuración de márgenes y encabezados: Establece los márgenes, encabezados y pies de página según la normativa de la SOE.
    \end{enumerate}

    Elementos específicos de la plantilla
    \begin{enumerate}
        \item Estilo de citación: Configura el estilo de citación utilizando paquetes como biblatex o natbib.
    Tipografía: Define la tipografía para el cuerpo del texto, los títulos, las secciones y otros elementos.
        \item Numeración de páginas: Configura la numeración de páginas (arábigos, romanos, etc.) y su ubicación.
    Tablas y figuras: Crea estilos para la presentación de tablas y figuras, incluyendo leyendas y referencias cruzadas.
        \item Código fuente: Incluye un estilo para la presentación de código fuente, resaltando la sintaxis y utilizando fuentes monoespaciadas.
        \item Matemáticas: Asegúrate de que la plantilla maneje correctamente las ecuaciones y símbolos matemáticos.
    \end{enumerate}
    
    Personalización y refinamiento
    \begin{enumerate}
    
        \item Colores y logotipos: Incorpora los colores y logotipos de la SOE.
        \item Fuentes personalizadas: Utiliza fuentes específicas si la unidad de posgrado lo requiere.
        \item Diseño de secciones: Define el diseño de las secciones, subsecciones y otros elementos jerárquicos.
        \item Ejemplos de uso: Incluye ejemplos de cómo utilizar la plantilla para diferentes tipos de documentos.
    \end{enumerate}
    
    Documentación y distribución
    \begin{enumerate}
        \item Manual de usuario: Crea un manual de usuario que explique cómo utilizar la plantilla y personalizarla.
        \item Plantilla en línea: Considera la posibilidad de crear una plantilla en línea en Overleaf para facilitar su uso y colaboración.
        \item Repositorio de la plantilla: Publica la plantilla en un repositorio (GitHub, GitLab, etc.) para que esté disponible para todos los miembros de la SOE. \url{https://github.com/profjcp/Articulos/}
    \end{enumerate}
    
    %the example \cite{fellows_research_2021}. 
    
    \subsection{Propuesta de plantilla base en Latex}
    Al adoptar LaTeX y utilizar la plantilla base propuesta, los investigadores de la unidad de posgrado SOE podrán producir documentos de alta calidad que cumplan con los estándares internacionales de publicación. 
    Esto no solo facilitará la difusión de sus investigaciones, sino que también contribuirá a elevar el prestigio académico de la unidad de posgrado.\\
    Plantilla ejemplo para la creación de artículos científicos en LaTeX:
    \begin{lstlisting}[style=mystyle]
      
        %%%%%%%%%%%%%%%%%%%%%%%%%%%%%%%%%%%%%%%%%%%%%%%%
        % @autor {Juan Carlos Peinado Pereira}
        % @correo {juanpeinado@uagrm.edu.bo}
        % @ano 2025
        % @version 1.0
        % @licencia CC BY 4.0
        %%%%%%%%%%%%%%%%%%%%%%%%%%%%%%%%%%%%%%%%%%%%%%%%
    
        % Se agradecen sus comentarios, contribuciones, y la difusion de esta plantilla
        
        % Tipo de documento y tamano de letra
        \documentclass[10pt]{article}
        
        % Ajuste de lenguaje
        \usepackage[spanish, english]{babel}
        
        % Tamano de pagina y margenes
        \usepackage[letterpaper, top=2.5cm,bottom=3.5cm, left=3cm, 
        right=2.5cm, marginparwidth=1.75cm]{geometry}
        
        % Permite manejo de ecuaciones y simbolos matematicos
        \usepackage{amsmath, amssymb, amsthm, amsfonts}
        
        % Permite el manejo de citas bibliograficas
        \usepackage{cite}
        
        % Permite manejo de la norma APA Apacite referencias bibliograficas
        \usepackage{apacite}
        
        % Paquete para agregar imagenes
        \usepackage{graphicx}
        
        % Fuente principal del texto: Times
        \usepackage{mathptmx}
        
        % Permite el manejo de caracteres especiales en el espanol
        \usepackage[utf8]{inputenc} 
        \usepackage[T1]{fontenc}
        
        % Subrayado de textos
        \usepackage[normalem]{ulem}
        
        % Personalizar los encabezados y pies de pagina
        \usepackage{fancyhdr}
        \pagestyle{fancy}
        
        % Ruta para imagenes
        \graphicspath{{image/}}
        
        % Ajusta la sangria
        \setlength{\parindent}{1cm}
        
        % Ajusta el espacio entre renglones
        \setlength{\parskip}{0.5cm}
        
        % Paquete para hipervinculos (debe ir despues de cite y apacite)
        \usepackage{hyperref}
        \hypersetup{
            colorlinks=true,
            linkcolor=black,
            urlcolor=black,
            citecolor=black
        }
        
        % Headers config
        % Cambiar Volumen, Numero y paginas
        \lhead{
            Fronteras Tecnologicas, Febrero 20, Vol.x, No.x, paginas \\ href:xx\\ISSN:x}
        \rhead{
            \includegraphics[width=1.2cm]{logo}}
        \headsep=60pt
        
        % Informacion del articulo
        \title{\Huge Titulo de investigacion\\}
        \author{\Large Autores de la publicacion}
        \date{\normalsize Datos de la Institucion de adscripcion de los autores.\\
        \textit{xxxx@soe.uagrm.edu.bo}\\
        Santa Cruz de la Sierra, Bolivia \\
        https://orcid.org/0009-0009-9117-2441}
        
        
        \begin{document}
        \maketitle
        \end{doocument}
        
    \end{lstlisting}

    %\begin{figure}[ht]
    %\centering
    %\makebox[\linewidth][c]{\includegraphics[width=1\linewidth]{Figures/PlantillaLatex.png}}%
    %\caption{Plantilla para Artículos Científicos}
    %label{fig-2}
    %\end{figure}
    
    Invitamos a los investigadores de la unidad de posgrado SOE a adoptar LaTeX y a utilizar la plantilla base proporcionada. 
    Para aquellos que deseen aprender a utilizar LaTeX, se ofrecerán un repositorio en GitHub con toda la documentación necesaria para iniciar en LaTeX. 
    Estamos seguros de que esta iniciativa contribuirá a mejorar la calidad y la eficiencia de la investigación en nuestra unidad de posgrado.  
    
Una vez que se ha aplicado la plantilla base propuesta en Latex en la ecritura de este documento, se obtienen los siguientes resultados:
    \begin{enumerate}
        \item Una alta calidad tipografía, con control preciso sobre la tipografía y el espaciado, tambien en la selección de funtes y alineados.
        \item Una estructura clara y consistente en todo el documento, con una variedad de plantillas y estilos predefinidos.
        \item Una gestión eficiente de citas con paquetes como biblatex y natbib, con una amplia variedad de estilos de citación.
        \item Una integración fluida de gráficos generados con herramientas externas, como TikZ o pgfplots.
        \item Una facilidad para la creación de tablas complejas con un código legible.
        \item Una facilidad para la escritura de ecuaciones complejas con un código legible.
        \item Una facilidad para la automatización de la generación de bibliografías y la inserción de citas.
    \end{enumerate}
Todas estas características hacen de LaTeX una herramienta valiosa para los investigadores de la unidad de posgrado SOE, como tambien para la escritura de este documento.\\
A continuación se presentan los resultados obtenidos en base a una comparativa realizada entre word vs LaTeX, después de haber aplicado el formato de "SOE" a la plantilla base propuesta en Latex.  \ref{tab:Comparativas}.

\vspace{0.5cm}
    \begin{table}[h!]
        \centering
        \resizebox{16cm}{!}{
            \begin{tabular}{|c|c|c|}
            \hline
            \textbf{Aspectos} & \textbf{Word} & \textbf{Latex} \\ \hline
            \textbf{Tipografía}      &         &                   \\ \hline
            {Calidad}         & Generalmente buena, pero limitada en opciones de personalización.          & Excelente, con control preciso sobre la tipografía y el espaciado.\\ \hline
            {Control}         & Limitado a las opciones predefinidas.          & Amplio control sobre todos los aspectos tipográficos.                  \\ \hline
            {Consistencia}    & Puede ser difícil mantener la consistencia en documentos largos.          & Garantiza la consistencia en todo el documento.                  \\ \hline
            \textbf{Formatos}        &         &                   \\ \hline
            {Predefinidos}    & Ofrece una variedad de plantillas y estilos predefinidos.          & Permite el uso de clases de documentos y paquetes para diferentes formatos.\\ \hline
            {Personalización} & Permite la personalización, pero puede ser complejo y llevar mucho tiempo.          & Ofrece una gran flexibilidad para la personalización.                \\ \hline
            {Estructura}      & Se basa en una interfaz WYSIWYG (lo que ves es lo que obtienes).          & Se basa en comandos y etiquetas para estructurar el documento.                  \\ \hline
            \textbf{Fuentes}         &         &                   \\ \hline
            {Variedad}        & Ofrece una variedad de plantillas y estilos predefinidos.          & Amplia variedad de fuentes disponibles, especialmente para matemáticas y símbolos.\\ \hline
            {Control}         & Control limitado sobre la selección y el uso de fuentes.          & Control preciso sobre la selección y el uso de fuentes.                \\ \hline
            \textbf{Citas bibliográficas}         &         &                   \\ \hline
            {Gestion}         & Ofrece herramientas para la gestión de citas, pero pueden ser limitadas.          & Ofrece una gestión eficiente de citas con paquetes como biblatex y natbib.                 \\ \hline
            {Estilo}          & Permite el uso de diferentes estilos de citación.          & Permite el uso de una amplia variedad de estilos de citación.\\ \hline
            {Automatización}  & Ofrece cierta automatización, pero puede requerir intervención manual.          & Automatiza la generación de bibliografías y la inserción de citas.                \\ \hline
            \textbf{Gráficos}        &         &                   \\ \hline
            {Creación}        & Permite la creación de gráficos básicos dentro del documento.          & Requiere el uso de herramientas externas (como TikZ o pgfplots) para generar gráficos de alta calidad.                \\ \hline
            {Personalización} & Ofrece opciones de personalización limitadas.          & Ofrece un alto grado de personalización para la apariencia de los gráficos.\\ \hline
            {Integración}     & Integración sencilla de gráficos creados en otras aplicaciones.          & Integración fluida de gráficos generados con herramientas externas.              \\ \hline
            \textbf{Tablas}          &         &                   \\ \hline
            {Creación}        & Permite la creación de tablas con una interfaz gráfica.        & Se crean mediante comandos y entornos.                \\ \hline
            {Formato}         & Ofrece opciones de formato básicas.          & Permite un control preciso sobre el formato y la apariencia de las tablas.\\ \hline
            {Complejidad}     & Puede ser complicado crear tablas complejas con múltiples filas y columnas.          & Facilita la creación de tablas complejas con un código legible.              \\ \hline
            \textbf{Expresiones matemáticas}          &         &                   \\ \hline
            {Escritura}        & Permite la inserción de ecuaciones con un editor de ecuaciones.        & Se escriben utilizando comandos y símbolos.                \\ \hline
            {Calidad}         & La calidad de las expresiones matemáticas puede ser limitada.         & Produce expresiones matemáticas de alta calidad con una tipografía profesional.\\ \hline
            {Complejidad}     & Puede ser complicado escribir ecuaciones complejas.          & Facilita la escritura de ecuaciones complejas con un código legible.              \\ \hline
            \end{tabular}
        }
        \caption{Comparativa entre Word y Latex, en la aplicacion de plantillas base.}
        \label{tab:Comparativas}
    \end{table}
\vspace{0.5cm}

    \section{Discusion}
    La adopción de LaTeX en la escritura de este documetno ha permitido mejorar significativamente la calidad tipográfica en la investigación presentada. 
    Lo que se traduce en una presentación más profesional y legible de los resultados. Sin embargo, la curva de aprendizaje inicial puede representar un desafío para algunos investigadores, lo que requiere la implementación de programas de capacitación y soporte técnico.
    Por este motivo es que se propone la creación de un repositorio en GitHub con toda la documentación necesaria para iniciar en LaTeX, así como la organización de talleres y seminarios para capacitar a los investigadores en el uso de esta herramienta.
    Para un futuro trabajo se propone la creación de una plantilla en línea en Overleaf para facilitar su uso y colaboración, así como la publicación de la plantilla en un repositorio (GitHub, GitLab, etc.) para que esté disponible para todos los miembros de la unidad de posgrado SOE.

    \section{Conclusiones}
    Asimismo, se concluye que LaTeX facilita la estructuración lógica y la organización eficiente de documentos extensos, un aspecto particularmente valioso en la investigación académica. 
    La capacidad de LaTeX para separar el contenido del formato permite a los investigadores concentrarse en el desarrollo de sus ideas, delegando en el sistema la tarea de dar forma al documento. 
    La generación automática de índices y tablas de contenido simplifica la navegación del lector a través del texto, mejorando la accesibilidad a la información.

%\nocite{*}
%\bibliographystyle{apacite}
\printbibliography
%\bibliography{referencias}

\end{document}